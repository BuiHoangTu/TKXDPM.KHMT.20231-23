\documentclass[../UseCaseSpecification.tex]{subfiles}

\begin{document}
\begin{enumerate}
    \item \textbf{Use case code}
    \newline
    UC003

    \item \textbf{Brief Description}
    \newline
    This use case describes the interaction between users and the AIMS software when the customer wishes to look for media items.

    \item \textbf{Actors}
    \begin{itemize}
        \item User
    \end{itemize}

    \item \textbf{Preconditions}
    \newline
    No

    \item \textbf{Basic Flow of Events}
    \begin{enumerate}
        \item The customer starts AIMS 
        \item AIMS displays 20 media items on each page 
        \item The customer looks for items 
    \end{enumerate}

    \item \textbf{Alternative flows}
    {\bfseries Table N-Alternative flows of events for UC Place order}
    \begin{flushleft}
        \tablefirsthead{}
        \tablehead{}
        \tabletail{}
        \tablelasttail{}
        \begin{supertabular}{|m{0.87200004cm}|m{1.918cm}|m{3.3cm}|m{5.803cm}|m{2.728cm}|}
            \hline
            \foreignlanguage{english}{\textbf{No}} &
            \foreignlanguage{english}{\textbf{Location}} &
            \foreignlanguage{english}{\textbf{Condition}} &
            \foreignlanguage{english}{\textbf{Action}} &
            \foreignlanguage{english}{\textbf{Resume location}} \\
            \hline
            
            \begin{enumerate}
                \item ~
            \end{enumerate} &
            At Step 3 &
            If the customer chooses to search items by their attributes &
            \begin{itemize}
                \item The customer selects attribute of media for searching, types in the attribute and presses \textit{search} button
                \item The AIMS 20 related media items on each searching page
            \end{itemize} &
            At Step 3 \\
            \hline

            \begin{enumerate}
                \item ~
            \end{enumerate} &
            At Step 3 &
            If the customer chooses to sort items by their price &
            \begin{itemize}
                \item The customer presses \textit{sort} button 
                \item The AIMS software displays the items in sorted order
            \end{itemize} &
            At Step 3 \\
            \hline
            
            \begin{enumerate}
                \item ~
            \end{enumerate} &
            At Step 3 &
            If the customer chooses to add item into cart &
            \begin{itemize}
                \item The customer types in wanted quantity and presses \textit{Add to Cart} button 
                \item The AIMS software adds items with the quantity into the cart
            \end{itemize} &
            At Step 3 \\
            \hline

        \end{supertabular}
    \end{flushleft}

    \item \textbf{Input data}
    {\bfseries  Table Input data of Search}
    \begin{flushleft}
        \tablefirsthead{}
        \tablehead{}
        \tabletail{}
        \tablelasttail{}
        \begin{supertabular}{|m{0.744cm}|m{2.181cm}|m{2.165cm}|m{2.052cm}|m{3.5939999cm}|m{4.086cm}|}
            \hline
            \foreignlanguage{english}{\textbf{No}} &
            \foreignlanguage{english}{\textbf{Data fields}} &
            \foreignlanguage{english}{\textbf{Description}} &
            \foreignlanguage{english}{\textbf{Mandatory}} &
            \foreignlanguage{english}{\textbf{Valid condition}} &
            \foreignlanguage{english}{\textbf{Example}}\\
            \hline
            
            \begin{enumerate}
                \item ~
            \end{enumerate} &
            Attribute selection box &
            selection box of available attributes of media &
            No (Default: Title) &
            ~ &
            Title \\
            \hline

            \begin{enumerate}
                \item ~
            \end{enumerate} &
            Attribute search box &
            ~ &
            Yes &
            Maximum of 50 characters &
            Falling Star \\
            \hline
            
        \end{supertabular}
    \end{flushleft}


\end{enumerate}
\end{document}